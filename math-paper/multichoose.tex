\documentclass[12pt]{article}
\usepackage{amsmath}
\usepackage[margin=1in]{geometry}

\newcommand{\multichoose}[2]{\left(\!\binom{#1}{#2}\!\right)}
\newcommand{\bigmultichoose}[2]{\left(\!\!{\binom{#1}{#2}}\!\!\right)}

\begin{document}

The following \LaTeX\ macros create commands for typesetting
the multichoose symbol:
\begin{verbatim}
\newcommand{\multichoose}[2]{\left(\!\binom{#1}{#2}\!\right)}
\newcommand{\bigmultichoose}[2]{\left(\!\!{\binom{#1}{#2}}\!\!\right)}
\end{verbatim}
Note that the \verb|\left| and \verb|\right| macros come from the
\verb|amsmath| package, so be sure to include
\verb|\usepackage{amsmath}| in your preamble. The macro \verb|\!| is a
``negative thin space'' that closes up the gap between the outer set
of parentheses and the binomial coefficient.

\begin{itemize}
\item Use the \verb|\multichoose| command for inline mathematics,
  e.g., $\multichoose{n}{k} = \binom{n+k-1}{k}$.

\item Use the \verb|\bigmultichoose| command for displayed
  mathematics:
  \[
  \sum_{k=0}^n \bigmultichoose{n}{k} .
  \]
\end{itemize}



\end{document}
